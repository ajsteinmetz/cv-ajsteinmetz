\documentclass[11pt]{article}
\usepackage[margin=1.0in]{geometry}
\usepackage{booktabs}
\usepackage[colorlinks=true,linkcolor=blue,citecolor=blue,urlcolor=blue]{hyperref}
\usepackage[shortlabels]{enumitem}
\usepackage{longtable}
\usepackage{graphicx}
\usepackage{subcaption}
\linespread{0.95} % Reduce overall line spacing
\setlength{\parskip}{0.05em} % Small parskip
%\setlength{\parindent}{0pt}

\begin{document}

\begin{center}
    {\Large\textbf{Andrew James Steinmetz, Ph.D.}}\\[0.5em]
    {\large\textbf{Teaching Statement}}
\end{center}

\noindent
My goal as an educator is to foster an intuitive understanding of the universe, both for myself and for my students. The mutual exchange of enthusiasm between myself and students creates a uniquely rewarding environment. I have been fortunate to teach students at the University of Arizona (UA), the Arizona College of Technology at Hebei University of Technology (ACT/HEBUT), and Pima Community College (PCC).

Students often grapple with personal anxiety, cultural differences, and unfamiliarity before fully engaging with the material. To overcome these barriers, I consistently present unshakable enthusiasm and a genuine interest in the subject matter. To keep student interest high, I tie theoretical topics to concrete examples or research data. I especially appreciate hearing from students when their perception of science has improved, a sentiment that resonates across all levels, from introductory courses to advanced courses.

\vspace{1em}

{\noindent\Large\textbf{Active Learning and Instructional Methods}}\\
My teaching style has evolved toward active learning, blending traditional lectures with in-class group activities. I organize students into small groups of three to five, encouraging them to work together repeatedly on challenges throughout the semester. I provide my students with detailed supplementary material such as relevant videos, images, computer programs, or data relevant to the topics being covered. This approach emulates the participatory nature of laboratory courses using pen-and-paper exercises, while also reinforcing teamwork skills essential for future careers. 

A modern challenge for instructors is the rise of AI tools like ChatGPT, which can solve even complex upper-level physics problems. To address this, I emphasize in-class evaluations and collaborative exercises to design assessments that remain meaningful and resilient to these technological shifts. I track student performance on exams by doing statistics on which topics they score well on versus those they have difficulty with and adjust my teaching throughout the year to compensate.

\vspace{1em}

{\noindent\Large\textbf{Diverse Teaching Experiences and Adaptability}}\\
My teaching journey spans diverse class sizes and student demographics, from smaller personalized classes at PCC to larger lecture and lab-based courses at ACT/HEBUT, where English is a second language for all students. During the COVID-19 pandemic, I was tasked with scripting and producing videos to explain and demonstrate undergraduate lab experiments. 

In addition to classroom teaching, I regularly advise students on graduate school applications and provide letters of recommendation. I initiated an undergraduate physics journal club at ACT to discuss advanced topics and research with highly motivated students.

\vspace{1em}

{\noindent\Large\textbf{Curriculum, Course Development, and Public Outreach}}\\
I have also had the opportunity to develop new courses at ACT/HEBUT. The UA physics curriculum includes two semesters of Experimental Methods in Physics (PHYS 381/382), an upper-level course that was not offered at HEBUT when I arrived in Spring 2024 to Tianjin, China. To implement these courses, I collaborated with laboratory managers, equipment vendors, and contractors (communicating mostly in Chinese via a translator) to refurbish a laboratory room and acquire equipment under tight deadlines. Designing these courses involved balancing theoretical rigor with practical skills, ensuring that experiments aligned with the UA curriculum while addressing the unique needs of ACT/HEBUT students. I also co-developed a Senior Design/Capstone course that emphasized career skills and group work in an engineering context. 

Outside the classroom, I enjoy sharing science with the public by uploading astrophotography to \href{https://www.astrobin.com/users/djinn/}{AstroBin}, discussing research on \href{https://bsky.app/profile/ajsteinmetz.com}{BlueSky}, and posting blog entries on my \href{https://ajsteinmetz.github.io/}{personal website}.

\end{document}
