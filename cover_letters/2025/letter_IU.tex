\documentclass[11pt]{article}
\usepackage[margin=1in]{geometry}
\usepackage{booktabs}
\usepackage[colorlinks=true,linkcolor=blue,citecolor=blue,urlcolor=blue]{hyperref}
\usepackage[shortlabels]{enumitem}
\usepackage{longtable}
\linespread{0.95}
\setlength{\parskip}{0.5em}
\setlength{\parindent}{0pt}

\begin{document}

% --- Cover Letter ---
Andrew J. Steinmetz\\
Department of Physics, The University of Arizona\\
Email: ajsteinmetz@arizona.edu\\
Phone: (520) 989-1305\\

April 2, 2025

\vspace{1em}

\noindent
Experimental Particle/Nuclear Postdoctoral Search Committee\\
Physics Department, Indiana University\\
Bloomington, IN

\vspace{1em}

\noindent
Dear Members of the Search Committee,

\vspace{1em}

I am writing to apply for the Postdoctoral Position in the Experimental Particle/Nuclear Physics group at Indiana University in Bloomington. I have a PhD in Physics from the University of Arizona, and a research background in theoretical relativistic spin dynamics, magnetism, and cosmology. Although my direct experience in hadron spectroscopy is limited, my collaborative work with researchers at institutions such as the University of Arizona, ELI-Beamlines ERIC, and Wigner RCP gives me a strong background of research success in a team environment.

My dissertation \textit{Modern Topics in Relativistic Spin Dynamics and Magnetism} discusses many topics related to magnetic moment dynamics, anomalous magnetic moments, and matter-antimatter interactions in quantum and cosmological contexts. This background provides me with a theoretical foundation, which I would like to use to contribute to experimental analyses, particularly those related to heavy quark mesons. I am eager to transition my theoretical expertise into experimental research efforts in hadron spectroscopy. Specifically, I have an interest in effective theories used to describe composite particles such as hadrons. My work on second-order fermion spin dynamics specifically highlights potential applications to describe composite particles.

I have experience both as a lead author and supportive co-author in publications in high-quality journals such as \textit{Physical Review A \& C} and \textit{European Physical Journal A \& C} and I am excited to pivot into publishing and grant-writing for experimental projects such as BESIII and GlueX. Additionally, my extensive teaching experience at the University of Arizona, Hebei University of Technology, and Pima Community College can be translated into public outreach skills for presentations and conference events requiring public speaking.

I have connections to the IU community as well, so I would be excited to move to the Bloomington area. I visited the IU campus in 2023 and met with Dr. Szczepaniak in the Physics Department and loved the beautiful campus, especially the chiming clock tower, and I would be thrilled to visit again. Thank you for considering my application.

\vspace{1em}

\noindent
Sincerely,

\vspace{2em}

\noindent
Andrew J. Steinmetz\\
Global Professor, Department of Physics\\
The University of Arizona

\clearpage

% --- Curriculum Vitae ---
\begin{center}
    {\Large\textbf{Andrew James Steinmetz, Ph.D.}}\\[0.5em]
    {\large\textbf{Curriculum Vitae}}
\end{center}

\noindent
\textbf{Email:} \href{mailto:ajsteinmetz@arizona.edu}{ajsteinmetz@arizona.edu}\\[0.3em]
\textbf{Phone:} (520) 989-1305\\[0.3em]
\textbf{Faculty Page:} \href{https://w3.physics.arizona.edu/person/andrew-steinmetz}{https://w3.physics.arizona.edu/person/andrew-steinmetz}\\[0.3em]
\textbf{Website:} \href{https://ajsteinmetz.github.io/}{https://ajsteinmetz.github.io/} (\textbf{GitHub:} \href{https://github.com/ajsteinmetz/}{ajsteinmetz})\\[0.3em]
\textbf{ORCID:} \href{https://orcid.org/0000-0001-5474-2649}{https://orcid.org/0000-0001-5474-2649}

\section*{Education}
{\normalsize
\begin{tabular}{@{}p{2.8cm} p{4.0cm} p{5.0cm} p{3.2cm}@{}}
\toprule
\textbf{Degree} & \textbf{Field} & \textbf{Institution} & \textbf{Dates} \\
\midrule
Ph.D. & Physics             & The University of Arizona & 2023 \\
B.S.   & Physics             & The University of Arizona & 2014 \\
B.S.   & Chemical Engineering& The University of Arizona & 2014 \\
\bottomrule
\end{tabular}
}

\section*{Employment}
{\normalsize
\begin{tabular}{@{}p{2.8cm} p{9.5cm} p{3.2cm}@{}}
\toprule
\textbf{Position} & \textbf{Institution} & \textbf{Dates} \\
\midrule
Global Professor   & Dept. of Physics, The University of Arizona            & {\footnotesize Nov.~2023--Present} \\
Global Professor   & Arizona College of Tech., Hebei University of Technology       & {\footnotesize Feb.~2024--Present} \\
Adjunct Faculty    & Physics \& Astronomy Dept., Pima Community College               & {\footnotesize June~2020--Dec.~2023} \\
\bottomrule
\end{tabular}
}

\section*{Research Interests}
Advancing relativistic spin dynamics and anomalous magnetic moments within quantum and classical frameworks. Researching the role of matter-antimatter interactions in cosmic magnetism. Developing novel asymptotic expansions for Fermi-Dirac integrals in extreme conditions in astrophysics, quark-gluon plasma, and cosmology. Additional interests include nonlinear electromagnetism, radiation reaction, and neutrino physics in strong fields. 

\section*{Dissertation}
\textbf{Title:} Modern Topics in Relativistic Spin Dynamics and Magnetism\\[0.3em]
\textbf{Committee:} Prof. Johann Rafelski~(Chair), Prof. Shufang Su~(Member), Prof. John Rutherfoord~(Member), Prof. Stefan Meinel~(Member), Prof. Sean Fleming~(Member)\\[0.3em]
\textbf{HDL:} \href{http://hdl.handle.net/10150/670301}{http://hdl.handle.net/10150/670301} \quad \textbf{Presentation:} \href{http://dx.doi.org/10.13140/RG.2.2.24323.27689}{10.13140/RG.2.2.24323.27689}\\[0.3em]

\newpage

\section*{Publications}
\subsection*{Peer-Reviewed Journal Articles}
\begin{enumerate}[leftmargin=*,nosep]
    \item \textbf{\href{https://github.com/ajsteinmetz/short-note-qgp}{Steinmetz, A.}}, Rafelski, J. ``Short Note on Spin Magnetization in QGP,'' \textit{Eur. Phys. J. ST} (in press, 2025), \href{https://doi.org/10.48550/arXiv.2502.05052}{arXiv:2502.05052}.
    \item Birrell, J., Formanek, M., \textbf{\href{https://github.com/ajsteinmetz/fermi-distribution}{Steinmetz, A.}}, Yang, C. T., Rafelski, J. ``Fermi-Dirac Integrals in Degenerate Regimes: Novel Asymptotic Expansion,'' \textit{Int. J. Theor. Phys.} 63, 163 (2024), \href{https://doi.org/10.1007/s10773-024-05695-8}{10.1007/s10773-024-05695-8}.
    \item Rafelski, J., \textbf{\href{https://github.com/ajsteinmetz/neutrino-transition-moments}{Steinmetz, A.}}, Yang, C. T. ``Dynamic fermion flavor mixing through transition dipole moments,'' \textit{Int. J. Mod. Phys. A} 38.31 (2023): 2350163, \href{https://doi.org/10.1142/S0217751X23501634}{10.1142/S0217751X23501634}.
    \item \textbf{\href{https://github.com/ajsteinmetz/plasma-partition}{Steinmetz, A.}}, Yang, C. T., Rafelski, J. ``Matter-antimatter origin of cosmic magnetism,'' \textit{Phys. Rev. D} 108 (2023): 123522, \href{https://doi.org/10.1103/PhysRevD.108.123522}{10.1103/PhysRevD.108.123522}.
    \item Formanek, M., \textbf{Steinmetz, A.}, Rafelski, J. ``Motion of classical charged particles with magnetic moment in external plane-wave electromagnetic fields,'' \textit{Phys. Rev. A} 103.5 (2021): 052218, \href{https://doi.org/10.1103/PhysRevA.103.052218}{10.1103/PhysRevA.103.052218}.
    \item Formanek, M., \textbf{Steinmetz, A.}, Rafelski, J. ``Radiation reaction friction: Resistive material medium,'' \textit{Phys. Rev. D} 102.5 (2020): 056015, \href{https://doi.org/10.1103/PhysRevD.102.056015}{10.1103/PhysRevD.102.056015}.
    \item Formanek, M., \textbf{Steinmetz, A.}, Rafelski, J. ``Classical neutral point particle in linearly polarized EM plane wave field,'' \textit{Plasma Phys. Control. Fusion} 61.8 (2019): 084006, \href{https://doi.org/10.1088/1361-6587/ab242e}{10.1088/1361-6587/ab242e}.
    \item \textbf{\href{https://github.com/ajsteinmetz/magnetic-dipole-moment}{Steinmetz, A.}}, Formanek, M., Rafelski, J. ``Magnetic dipole moment in relativistic quantum mechanics,'' \textit{Eur. Phys. J. A} 55, 40 (2019), \href{https://doi.org/10.1140/epja/i2019-12715-5}{10.1140/epja/i2019-12715-5}.
    \item Formanek, M., Evans, S., Rafelski, J., \textbf{Steinmetz, A.}, Yang, C. T. ``Strong fields and neutral particle magnetic moment dynamics,'' \textit{Plasma Phys. Control. Fusion} 60.7 (2018), \href{https://doi.org/10.1088/1361-6587/aac06a}{10.1088/1361-6587/aac06a}.
    \item Rafelski, J., Formanek, M., \textbf{Steinmetz, A.} ``Relativistic dynamics of point magnetic moment,'' \textit{Eur. Phys. J. C} 78 (2018): 1--12, \href{https://doi.org/10.1140/epjc/s10052-017-5493-2}{10.1140/epjc/s10052-017-5493-2}.
\end{enumerate}

\subsection*{Book Chapters \& Review Articles}
\begin{enumerate}[leftmargin=*,nosep]
    \item Rafelski, J., Birrell, J., Grayson, C., \textbf{\href{https://github.com/ajsteinmetz/thesis-collab-project}{Steinmetz, A.}}, Yang, C. T. ``Quarks to Cosmos: Particles and Plasma in Cosmological evolution,'' \textit{Eur. Phys. J. ST} (in press, 2025), \href{https://doi.org/10.48550/arXiv.2409.19031}{arXiv:2409.19031}.
    \item Rafelski, J., \textbf{Steinmetz, A.}, Yang, C. T. ``Dynamic Flavor Mixing Through Transition Moments,'' \textit{Harald Fritzsch Memorial Volume}, pp. 269--284 (2024), \href{https://doi.org/10.1142/9789811292279_0015}{10.1142/9789811292279\_0015}.
    \item Rafelski, J., Birrell, J., \textbf{\href{https://github.com/ajsteinmetz/a-short-survey}{Steinmetz, A.}}, Yang, C.T. ``A Short Survey of Matter-Antimatter Evolution in the Primordial Universe,'' \textit{Universe} 9.7 (2023): 309, \href{https://doi.org/10.3390/universe9070309}{10.3390/universe9070309}.
\end{enumerate}

\subsection*{Works in Progress}
\begin{enumerate}[leftmargin=*,nosep]
    \item \textbf{Steinmetz, A.}, Evans, S., Formanek, M., Grayson, C., Labun, L., Price, W., Rafelski, J. ``Strong fields in classical and quantum physics,'' (in preparation, 2025).
    \item \textbf{Steinmetz, A.}, Rafelski, J. ``Magnetized primordial heavy-quark plasma,'' (in preparation, 2025).
    \item \textbf{Steinmetz, A.}, Yang, C. T., Rafelski, J. ``Electromagnetic field forcing of dynamic CP-violation in lepton sector,'' (in preparation, 2025).
    \item Evans, S., \textbf{Steinmetz, A.} ``Anomalous magnetic moment in the QCD vacuum,'' (in preparation, 2025).
\end{enumerate}

\subsection*{Public Outreach}
\begin{enumerate}[leftmargin=*,nosep]
    \item \textbf{Steinmetz, A.} (2024, November 7). Proving that SU(2) is compact (and other group theory bits). \href{https://ajsteinmetz.github.io/mathematics/2024/11/07/su2-compactness.html}{https://ajsteinmetz.github.io/mathematics/2024/11/07/su2-compactness.html}
    \item \textbf{Steinmetz, A.} (2024, October 17). Einstein’s mass-energy and kinetic energy.\\ \href{https://ajsteinmetz.github.io/physics/2024/10/17/kinetic-energy-coefficient.html}{https://ajsteinmetz.github.io/physics/2024/10/17/kinetic-energy-coefficient.html}
    \item \textbf{Steinmetz, A.} (2024, October 16). Can we ever detect the graviton?\\ \href{https://ajsteinmetz.github.io/physics/2024/10/16/graviton-detector.html}{https://ajsteinmetz.github.io/physics/2024/10/16/graviton-detector.html}
\end{enumerate}

\section*{Conference Presentations \& Departmental Talks}
\begin{enumerate}[leftmargin=*,nosep]
    \item \textbf{Title:} Primordial Cosmic Magnetism. \textbf{Event:} SMT 30th Anniversary Steward Observatory Symposium. The University of Arizona, Tucson, Arizona. September 22, 2023,\\ \href{http://dx.doi.org/10.13140/RG.2.2.19213.24806}{10.13140/RG.2.2.19213.24806}.
    \item \textbf{Title:} Magnetism in the Cosmic Plasma Epoch. \textbf{Event:} ELI-Beamlines Strong Fields Frontiers. The Extreme Light Infrastructure ERIC, Prague, Czech Republic. June 13, 2023,\\ \href{http://dx.doi.org/10.13140/RG.2.2.32635.02087}{10.13140/RG.2.2.32635.02087}.
    \item \textbf{Title:} Magnetism in the Cosmic Plasma Epoch. \textbf{Event:} Margaret Island Symposium on Particles and Plasmas. HUN-REN Wigner Research Centre for Physics, Budapest, Hungary. June 8, 2023, \href{http://dx.doi.org/10.13140/RG.2.2.29279.57762}{10.13140/RG.2.2.29279.57762}.
    \item \textbf{Title:} Relativistic Two-Body Quantum Mechanics. \textbf{Event:} Physics Department Grad Talk. The University of Arizona, Tucson, Arizona. March, 2017.
\end{enumerate}

\section*{Teaching Experience}
\textbf{Faculty Courses at UA/ACT (as Instructor-of-Record)*:}\\
{\normalsize
\begin{tabular}{@{}p{2.2cm} p{6.0cm} p{2.0cm} p{2.0cm} p{2.8cm}@{}}
\toprule
\textbf{Course \#} & \textbf{Title} & \textbf{Sections} & \textbf{Students} & \textbf{Semester} \\
\midrule
PHYS 371   & Quantum Theory                    & 1 & 75  & Spring 2025 \\
PHYS 381   & Methods in Exp. Physics I         & 2 & 67  & Spring 2025 \\
PHYS 321   & Theoretical Mechanics             & 2 & 139 & Fall 2024 \\
PHYS 382   & \textbf{Methods in Exp. Physics II**}       & 2 & 46  & Fall 2024 \\
ENGR 498A  & \textbf{Senior Design/Capstone**}           & 1 & 43  & Fall 2024 \\
PHYS 240   & Intro. Electricity \& Magnetism    & 3 & 203 & Spring 2024 \\
PHYS 381   & \textbf{Methods in Exp. Physics I**}        & 2 & 50  & Spring 2024 \\
\bottomrule
\end{tabular}
}

\medskip

* Recent student course evaluations available upon request.\\[0.0em]
** First time offered at this institution. Designed, developed, and delivered curriculum.\\

\textbf{Adjunct Courses at PCC (as Instructor-of-Record):} Intro. Physics I/II (2020, 2021, 2023), The Solar System (2020--2023), Intro. Electricity \& Magnetism (2022)

\textbf{TA Courses at UA:} Intro. to Scientific Computing (2017), Intro. Electricity \& Magnetism (2017), Methods in Exp. Physics I/II (2018, 2019), Intro. Mechanics I/II Lab (2020), Intro. Physics I/II Lab (2020), General Chemistry I(Honors)/II Lab (2019, 2021--2023)

\textbf{Key:} UA (The University of Arizona), ACT (Arizona College of Tech., Hebei University of Technology), PCC (Pima Community College)

\section*{Academic Service}
\begin{itemize}[leftmargin=*,nosep]
    \item Organizer, Undergraduate Journal Club, Hebei University of Technology, 2025.
    \item Member, Department of Physics, University of Arizona, Faculty Search Committee, 2025.
    \item Member, Department of Physics, University of Arizona, Faculty Search Committee, 2024.
    \item Reviewer for \textit{Physical Review D}, \textit{Modern Physics Letters A}, \textit{MDPI Universe}, and \textit{ICPMS 2025}.
\end{itemize}

\section*{Press}
\begin{enumerate}[leftmargin=*,nosep]
    \item World Scientific Publishing. (2025, January 9). \textit{World Scientific Publishing on LinkedIn: The International Journal of Modern Physics A (IJMPA) celebrates 40 Years...} LinkedIn.\\ \href{https://www.linkedin.com/embed/feed/update/urn:li:share:7282593891487293440}{https://www.linkedin.com/embed/feed/update/urn:li:share:7282593891487293440}
    \item Kong, X. (2024, March 20). Prof. Andrew Steinmetz of the University of Arizona Visited the Physics Demonstration and Exploration Lab. \textit{Physics Experiments at HEBUT, Hebei University of Technology.}
    \item The University of Arizona, Department of Physics. (2024, January 26). \textit{Announcing Department of Physics new Faculty Member Prof. Andrew Steinmetz}. UA Science Physics.\\ {\footnotesize\href{https://w3.physics.arizona.edu/news/announcing-department-physics-new-faculty-member-prof-andrew-steinmetz}{https://w3.physics.arizona.edu/news/announcing-department-physics-new-faculty-member-prof-andrew-steinmetz}}
    \item The University of Arizona, Department of Physics. (2023, July 25). \textit{International Symposium on "Particles and Plasmas" and "Strong Fields."} UA Science Physics.\\ {\footnotesize\href{https://w3.physics.arizona.edu/news/international-symposium-particles-and-plasmas-and-strong-fields}{https://w3.physics.arizona.edu/news/international-symposium-particles-and-plasmas-and-strong-fields}}
    \item Springer. (2018, January 29). Relativity matters: Two opposing views of the magnetic force reconciled. \textit{Phys.org}. \href{https://phys.org/news/2018-01-relativity-opposing-views-magnetic.html}{https://phys.org/news/2018-01-relativity-opposing-views-magnetic.html}\\[0.3em]
    Also published in: \href{https://www.epj.org/epjc-news/1422-epjc-highlight-relativity-matters-two-opposing-views-of-the-magnetic-force-reconciled}{EPJ C Highlight}, \href{https://www.springer.com/gp/about-springer/media/research-news/all-english-research-news/relativity-matters--two-opposing-views-of-the-magnetic-force-reconciled/15417658}{Springer Press}, \href{https://www.sciencedaily.com/releases/2018/01/180129131327.htm}{ScienceDaily}, EuroPhysicsNews, EurekAlert!, Science Newsline, Sky Nightly, Space Daily.
\end{enumerate}

\section*{Links \& Websites}
\begin{itemize}[leftmargin=*,nosep]
    \item \textbf{Inspire-HEP:} \href{https://inspirehep.net/authors/1796313}{https://inspirehep.net/authors/1796313}
    \item \textbf{Google Scholar:} \href{https://scholar.google.com/citations?user=fJBK1GIAAAAJ}{https://scholar.google.com/citations?user=fJBK1GIAAAAJ}
    \item \textbf{arXiv:} \href{https://arxiv.org/a/steinmetz\_a\_1.html}{https://arxiv.org/a/steinmetz\_a\_1.html}
    \item \textbf{BlueSky:} \href{https://bsky.app/profile/ajsteinmetz.com}{https://bsky.app/profile/ajsteinmetz.com}
    \item \textbf{AstroBin:} \href{https://www.astrobin.com/users/djinn/}{https://www.astrobin.com/users/djinn/}
    \item \textbf{LinkedIn:} \href{https://www.linkedin.com/in/ajsteinmetz/}{https://www.linkedin.com/in/ajsteinmetz/}
    \item \textbf{My Erd\H{o}s number is 5.} \href{https://mathscinet.ams.org/mathscinet/freetools/collab-dist?source=1443426\&target=189017}{(Source)}
\end{itemize}

\section*{Professional References}
{\normalsize
\begin{itemize}[leftmargin=*,nosep,label={\textbullet}]
    \item \textbf{Prof. Johann Rafelski} -- Professor of Physics, University of Arizona.\\
    Email: \href{mailto:johannr@arizona.edu}{johannr@arizona.edu} \quad Phone: (520) 777-9519
    \item \textbf{Prof. Shufang Su} -- Department Head, Professor of Physics, University of Arizona\\
    Email: \href{mailto:shufang@arizona.edu}{shufang@arizona.edu} \quad Phone: (520) 621-5540
    \item \textbf{Prof. Srin Manne} -- Associate Professor of Physics, University of Arizona\\
    Email: \href{mailto:smanne@physics.arizona.edu}{smanne@physics.arizona.edu} \quad Phone: (520) 626-5305
    \item \textbf{Prof. Ajith Rajapaksha} -- Global Professor, University of Arizona\\
    Email: \href{mailto:ajithr@arizona.edu}{ajithr@arizona.edu}
    \item \textbf{Dr. Martin Formanek} -- Marie Sklodowska-Curie Fellow, ELI-Beamlines ERIC\\
    Email: \href{mailto:martin.formanek@eli-beams.eu}{martin.formanek@eli-beams.eu} \quad Phone: (520) 248-6053
\end{itemize}
}

\end{document}
