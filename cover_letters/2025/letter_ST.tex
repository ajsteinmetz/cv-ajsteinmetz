\documentclass[11pt]{article}
\usepackage[margin=1.0in]{geometry}
\usepackage{booktabs}
\usepackage[colorlinks=true,linkcolor=blue,citecolor=blue,urlcolor=blue]{hyperref}
\usepackage[shortlabels]{enumitem}
\usepackage{longtable}
\linespread{0.95} % Reduce overall line spacing
\setlength{\parskip}{0.25em} % Small parskip
%\setlength{\parindent}{0pt}

% Position Information

\newcommand*{\PositionName}{Visiting Assistant Professor of Physics}
\newcommand*{\DepartmentName}{Physics Department}
\newcommand*{\FullName}{University of St. Thomas}
\newcommand*{\ShortName}{St. Thomas}
\newcommand*{\StreetAddress}{2115 Summit Avenue}
\newcommand*{\CityState}{Saint Paul, Minnesota 55105}

% Cover Letter

\begin{document}

% Address

\noindent
\begin{tabular}{@{}p{7.0cm} p{9.0cm}@{}}
    & Global Prof. Andrew J. Steinmetz\\
    & Department of Physics, The University of Arizona\\
    & 1118 E. Fourth Street, Tucson, AZ 85721\\
    & \href{mailto:ajsteinmetz@arizona.edu}{ajsteinmetz@arizona.edu} \(\vert\) (520) 989-1305
\end{tabular}

\vspace{1em}

\noindent
April 9, 2025

\vspace{1em}

\noindent
\PositionName\ Search Committee\\
\DepartmentName, \FullName\\
\StreetAddress, \CityState

\vspace{2em}

% Main Letter Body

\noindent
Dear Members of the Search Committee,

\noindent
I am writing to apply for the \PositionName\ position at the \FullName. I have a PhD in Physics from the University of Arizona (UA), and a research background in theoretical relativistic spin dynamics, magnetism, and cosmology. I have extensive experience teaching physics at the undergraduate level as well as managing laboratory courses, therefore, I believe I am an excellent candidate for this position.

\indent My current role is Global Professor at UA with a joint appointment in the Arizona College of Technology (ACT), Hebei University of Technology (HEBUT) in Tianjin, China. After being hired as a Global Professor at UA, I went on to teach intermediate to upper division physics courses (i.e. theoretical mechanics, quantum mechanics). I'm active in student mentoring as well as the organizer for the physics journal club at ACT. Prior to my current position at the UA, I was an adjunct faculty member at Pima where I was the instructor-of-record for introductory astronomy and physics courses managing both the lecture and laboratory components of the courses.

At ACT/HEBUT, I successfully designed and implemented a two-semester advanced undergraduate physics laboratory curriculum, coordinating closely with laboratory managers, equipment vendors, and contractors (communicating mostly in Chinese via a translator) to refurbish a laboratory room and acquire equipment under tight deadlines. I am proud to say we now have a fully functional multipurpose undergraduate laboratory for our ACT students. I also co-developed the Senior Design/Capstone course that emphasized career skills and group work in an engineering context. Our first graduates of the program will earn their degrees this June, 2025. Given all of these skills I've developed in my career so far, I am excited to bring my expertise to \ShortName.

In addition to my teaching qualifications, I have continued to advance my research profile with regular publications in cosmology, strong field physics, and statistical mechanics. Two recent publications are Steinmetz \& Rafelski ``Short Note on Spin Magnetization in QGP'' \emph{Eur Phys J ST} (in press, 2025) and Birrell et al. ``Fermi-Dirac Integrals in Degenerate Regimes: Novel Asymptotic Expansion'' \emph{Int J Theor Phys} (2024). The first work outlines a theory of cosmic magnetism in the early universe based on quark-gluon plasma magnetization while the second explores novel expansions of the Fermi-Dirac distribution for degenerate gases. I look forward to continuing my research at \ShortName\ as well as collaboration with faculty and students.

Thank you for your consideration and I look forward to hearing back from the committee.

% Closing

\vspace{1em}

\noindent
Sincerely,

\vspace{3em}

\noindent
Prof. Andrew J. Steinmetz\\
Department of Physics, The University of Arizona

\end{document}
